\section{Introduction}

% Background of the project, 
Student wellbeing is defined as a sustainable state of positive mood and attitude, resilience, and satisfaction with self, relationships and experiences
at school \cite{noble2008scoping}.
Most disorders that pertain to mental health first start developing between the ages of 15 and 20 years of age \cite{kessler2005lifetime}.
Roughly 15\% to 20\% of this demographic \cite{kessler2005lifetime} are suggested to end up becoming affected by some mental disorder; mostly 
around anxiety.
With universities introducing many new potential stresses for students, it is essential that those that attend have access to the appropriate help
and resources.
From statistics released by UCAS \cite{ucas_2018}, around 353,960 students in the UK moved on to pursue further education.
After applying the 15\% statistic, being on the conservative end of the scale, roughly 53,000 students are likely to go into their first year of
university with existing mental health issues.

% Motivation 
The motivation for this project stems from a desire to create an easy to use application that can help track a student's wellbeing and provide them 
with necessary resources based on that data.
Based on the initial idea and research covered in the project initiation document (Appendix \ref{appendix:pid}), a questionnaire/survey tool will 
be made to help address the issue around student wellbeing.
Using such a tool makes sense due to it being  easier to collect response data around a particular topic and scope which can be defined by the user
that creates the survey.
When looking at existing software solutions that allow a user to create custom questionnaires/surveys, it becomes apparent that they are specifically 
designed to appeal to a large number of use cases.
Most of the time they are used to collect data for marketing purposes, so have specific tools are built in to help with that.
For a survey tool to be effective amongst those that may be seeking some advice, a tailored solution will be required.

% Aims and objectives
As a software package, this project aims to provide an intuitive solution that allows mental health experts to curate a set of surveys that students
can respond to.
Core components of the project will include creating a server component that stores the relevant data into a database.
Client-side applications that can be run from both a web browser and an Android application also need to be created so that users can easily use the applications from a computer and smartphone device (respectively).
Data retrieved from students responding to surveys can then be utilised to help provide relevant resources to them.

% Summary of contributions and achievements
This report aims to guide the reader through the existing problem of student wellbeing and creating a set of requirements and constraints based on that.
The requirements influence the topics that need to research for implementing the desired solution along with the approach taken for the 
architecture and implementation.