\section{Evaluations}

This section aims to be a reflection on the work that has been carried out over the course of the entirety of this project.

\subsection{Application evaluation}

\subsubsection{Successes of the application}
Considering the complexities involved with this project, what has been implemented has been a success.
The project has managed to cover the main groundwork with regards to both the server and client side of the overall 
software solution.


\subsubsection*{Scalability}
One of the main requirements for the system was for it to support future scalability.
Completely separating both the front and and back end components allows each layer to consist of multiple instances of the same application as 
a form of load balancing.

Adopting a session-less approach, with Oauth2 and JWT, means that many servers can be deployed at once without having to share user sessions.
Having no sessions stored server side removes scalability issues as the memory capacity of a server will end up being a limiting factor.
Tokens are just issued by an authorization server and are stored client side instead.
It is also much easier to perform horizontal scaling as no data needs to be shared between each server instance.
Another benefit is that there is no recovery required if a server needs to be restarted.

\subsubsection*{Security}
An emphasis was made on security for the application so time that could have been spent on features was spent on making the application more secure.
Developing a secure set of API endpoints, from the ground up, specifically for the student wellbeing application, has been a great success.
The problem was that this API is no good in a real world environment if any type of user can have access to resources that should be 
The developer was able to successfully incorporate their own Oauth2 implementation to provide session-less authentication where a client is given
a Json Web Token to access restricted resources.
Any registered user must provide valid credentials before they are issued an access token and will only be valid for a limited period of time.
Tokens that are issued have an associated token secret which only the server will know of, this is another method that is used to ensure requests are 
being made with a legitimate token.

\subsubsection*{Usability}
The survey creation tool, that has been fully implemented, managed to pass the user acceptance testing stage which serves as proof of 
the sound functionality.
Even though there was no upfront design created for the client application, what has been developed is clearly presented and logically laid
out.
The components that make up each of the pages have been created in a way that ensures they behave the way users expect them to.
It would really frustrate the user if the application had not been updating other page components based on their own interactions.
Utilising Twitter's Bootstrap library has allowed the application to have a reactive layout; adapting to the screen size of the device being used.
In the absence of any mobile native application implementation, users can still access and use the web application as intended.

\subsubsection{Features not implemented}
Unfortunately due to the time constraints involved, the developer has not been able to implement a number of feature.
Most of these features related to the student usage of the application have not been implemented.

\subsubsection*{User registration}
Being able to register new users to the system is obviously an important part to the overall application. 
Before being able to handle user registrations, a secure method of authenticating users was considered a bigger priority.  
At the moment, the current implementation automatically \emph{registers} a test user when the user information in the database is detected to be 
empty.
With more time, a set of user registration API endpoints would have been created to support this functionality as the main groundwork has already 
been achieved.

\subsubsection*{Survey response tool}
The other major component of the application that was not included as a part of this release is the ability to respond to published surveys.
This is mostly because of the complexities involved with the survey creation tool and it did not make sense to start with a survey response tool
while there was no way to create them in the first place.

\subsubsection*{Android application}
The initial requirements stated that an android application needs to be provided as a part of the software solution.
A design decision was made to use the REACTjs library for developing the clint side application, the main motivation for this was that it would a 
lot easier to convert the existing code into REACT Native code which is used to create native applications for both Android and IOS.
If more time was available, then being able to learn how to convert the code and actually create such an application would have been possible.
The best alternative achieved that was achieved was the web application being fully usable on smart phones due to the taking advantage of
the Bootstrap library.

%--------------------------------------------------An emphasis was made on security for the application so time that could have been spent on features was spent on making the application more secure.-----------------------------------------------------------%

\subsection{Social, legal \& ethical aspects}
Social, legal and ethical concerns need to be addressed before this system can be deployed for public use.

\subsubsection{Social}

Regarding social aspects, this application revolves around the mental health of students within a university environment.
The idea behind it is to provide resources to those individuals that find it difficult to engage in the initiative of going to their general practitioner
(GP) to get assessed.
This software solution is meant to be a way to give users the anonymity that they may need to participate in something that can help improve 
various 
Hopefully it will allow those attending university to improve results in assessments throughout the year and achieve a good degree level.
Students can just use the application in a more informal way, utilising it to track the overall wellbeing and be able to compare it to other peers that
they interact with.


\subsubsection{Legal}
Personally identifiable information, of users, will be stored in datbases for this application to function.
Such identifiable information will include names, email addresses and data collected related to that person's wellbeing status.
Under the General Data Protection Regulation (GDPR), such information needs to be processed lawfully, fairly and in a transparent manner \cite{unihighlands2018sevenprinciples}.
The data itself must only be collected and used for legitimate purposes, in the case of this application it should only be used to calculate wellbeing 
scores and have that data saved in a user created account.
Data that is associated with a particular user needs to be removed from all parts of the system once the data is no longer needed or if the user makes
an explicit request.
Extra precautions should be taken to ensure that breaches do not occur to the database system and have that data leaked into the public domain.

This also can be paired with the Data Protection Act (1998) where there are eight conditions to comply when storing data.
These eight conditions include points that include keeping data secure, relevant and accurate for the relevant amount of time. 
Ensuring that data is processed in a lawful manner is also something covered under the act.

Before deploying the application into a full production environment where users will be able to use it, a full report will need to be generated that
outlines how well the overall system complies with these laws.
The involvement of data protection officers (DPO) is a fundamental part of GDPR. 
They ensure that compliance is met with the requirements set out by GDPR.

\subsubsection{Ethical}
No surveys were conducted as a part of this project, so no ethics approval was needed for this project.
In the future when this prototype is being developed further into a system suitable for a production environment, ethics approval will be requested
to gather user feedback about topics that pertain to the initial problem statement.

%-----------------------------------------------------------------------------------------------------------------------------%
\subsection{Developer performance}

There were many firsts for the developer in this project as this is the first time they have developed a web application.
To understand the type of skills and knowledge revolving the technologies used in this project, an evaluation needs to be conducted.

%-----------------------------------%

\subsubsection{Challenges}

\subsubsection*{Initial struggles with focus} \label{focusissues}
During the initial stages of the project, the developer was finding it very difficult to focus on the initial planning and prototyping stages.
This often meant that not enough progress was being made during the first few months of the academic year, leading to a lot of wasted time.
Thankfully due to the support given by their supervisor, they were able to add much needed limits the scope for the project and get the required 
focus to deliver frequent demonstrations.

\subsubsection*{SpringBoot}
Though the developer had some basic prior experience with the Java Spring framework, there were still many challenges to overcome 
in order to get the necessary features implemented.
A lot of the issues revolved around the particular \emph{Spring way} of configuring the application.
Akin to how someone with Java experience may want to start developing with Android, there is an entire structure and set of libraries
that need to be understood before anything can truly be created.

One of the biggest challenges with Spring was regards to the security configuration.
Understanding how technologies such as Ouath2 worked was the first hurdle but the main issue was attempting to implement that into the project.
While researching for the correct way to implement it, there proved to be many outdated or incorrect examples. 
The developer should have instead just referred to official guides rather than relying on these examples that are commonly posted on forums
such as StackOverflow.
This is one of the issues when working with technologies that are under constant development due to the growing demand.

\subsubsection*{Using Swagger2}
A lot of time was spent attempting to incorporate Swagger into the project as it was something they had learnt while on industrial placement.
The problem with this was that the developer did not properly understand the usage of the technology and how it is meant to be incorporated.
It was only after discussion with a former colleague that the understood how it is meant to be used which led to them determining that it did 
not fully meet their needs. 
The idea behind it was that you create the API upfront and then generate Java interfaces from it, these interfaces can then have implementations.
Instead, the approach taken was to develop the API directly in the Java code and use the documentation interface feature for keeping track of
all the API endpoints created.
The API could then be exported in JSON format and imported into testing tools such as Postman.

\subsubsection*{JavaScript and REACT}

Unlike Java (and Spring), the developer has never used JavaScript or REACT to develop an application that can be used from a web browser 
environment.
This posed many issues at the start as there was little clarity on how to actually implement functionalities such as the survey editor.
Coding techniques for implementing asynchronous calls proved to be challenging at first due to the lack of understand of how they worked within the 
JavaScript language.
Once the initial learning stage had passed, and a good foundation was created, a sense of momentum started to build; being able to implement new 
functionalities every week.

\subsubsection*{Test Driven Development}

Unfortunately a test driving approach to the development of the application was not done as intended.
This is due to how the author was learning many of the technologies while developing, so it made it hard for them to actually write proper tests
before implementing a feature.
Learning technologies such as Mockito to mock instances of classes for testing Java Spring components is needed.
Mockito in itself is not a very straightforward topic and was taking time away from actually developing features.

Similarly with REACT, many technologies were being learnt while developing the front end application.
This led to a lack of clarity of what is required for a TDD based approach with this type of code.

Ultimately manual testing, while developing, was adopted rather than proper unit testing.
Postman, an API testing tool, was used to manually test each of the REST endpoints to ensure that they were behaving as expected and were producing
the correct output.
Manual testing in a web browser was done while developing the REACT client to check if any forced errors caused a failure.

%-----------------------------------%

\subsubsection{Strengths}

\subsubsection*{Taking the view of a user}
It was important to the developer that the application ends up having a well thought out design approach.
Before implementing any feature, they took time to think about how this would affect the user's experience. 
This lead to more time being spend on how the client application function, implementing functionality that would likely be seen in more professional
applications.

\subsubsection*{Taking industry practices into account}
Many of the choices that relates to the choices in design and technologies came down to what the developer thought would make most sense if this application
was to made for a real client.
This is mostly due to the experiences gained from the time they had spent on placement.
Using technologies such as SpringBoot was heavily based on the exposure they had to the technology on their industrial placement.

\subsubsection*{Time management}
Once the challenge mentioned in section \ref{focusissues} had been overcome, the author was able to manage time very well.
They utilised a Kanban agile approach for the entirely of the development which meant that they focused on completing individual tasks before starting
another one. 
With a number of modules setting up assignments throughout the year, time had to be correctly spent to ensure that enough focus was still being maintained 
on this project.
Planning was done on a weekly basis and was typically agreed by both the developer and supervisor in their weekly catch up sessions. 

\subsubsection*{Willingness to be challenged}
Rather than going with relatively easy approaches, such as using PHP for the server, the developer wanted to learn the best technologies for modern web applications.
This often meant that more complex approaches, such as using Java Spring, were chosen over other solutions.
A focus was made therefore made on trying to achieve the best implementation possible for as many features as possible, even if it meant that all components 
could not be implemented on time.

%-----------------------------------%

\subsubsection{Skills gained}

\subsubsection*{Technologies}
The developer has been exposed to a vast number of different technologies throughout this development.
Just taking the server side application into account the following technologies have been used:
\begin{itemize}
    \tightlist
    \item Java
    \item SpringBoot (Spring framework)
    \item Hibernate
    \item Swagger2 (OpenAPI2.0)
    \item REST
    \item Oauth2 
    \item JWT
    \item Gradle (Build tool)
\end{itemize}

Modern web security practices were learnt and utilised in the project which the developer found satisfaction with.
Oauth2 with Json Web Tokens is now widely used across the web and having the motivation to implement such modern techniques made the developer aware 
of how they work.
That base knowledge allowed them to understand the Spring documentation of configuring the project correctly to add support for this.
Moving forward, they should be able to easily 

With regards to the font end client, there were plenty of technologies used.
The full list of that were used for the client are: 
\begin{itemize}
    \tightlist
    \item JavaScript
    \item JSX
    \item REACT
    \item Boostrap
    \item Yarn (Build tool)
\end{itemize}

REACT proved to be a very simple way of creating well structured code for something the user will actually interact with.
Everything is represented as components that can be individually refreshed depending if their states were updated or not.
Twitter's Bootstrap library was also incorporated to provide an easy approach to creating well presented pages to the user without having to spend time
toying around with CSS.

All of these technologies that were used have proven to be a great choice for web application that are to be used in the modern day incarnation of the
internet. 
The developer is sure that they are likely to utilise many of these technologies in future projects, moving forward.


\subsubsection*{Project management}
Due to this being a one person project, the project management side was also a responsibility for the developer.
They have been able to mananage from the initial idea stages to making up the initial designs and carrying through with the implemntations.
Typically this is not the case in the real worlds when these different responsibilities can be assigned to different people who are specialised in 
those areas.
Being forced to be responsible with the imposed time limitations has led to a better understanding of what it takes to plan out and follow through
with an entire project.  
An important part of this was the need to have reviews of the overall progress being made at regular intervals.
This is where there the regular meetings arranged with their project supervisor were very useful at it allowed an agreement to be made on what should 
be achieved by the next meeting.
This supported the idea behind using an agile approach to development where progress is being evaluated at regular intervals; changing direction if 
needed.

\subsubsection*{Evaluation skills}
A result of having to write this report, the author has been able to reflect on the overall effectiveness and success of the project.
Evaluations also had to be conducted on all of the potential technologies that could have been used and had to argue why the technologies chosen 
were the best approach.