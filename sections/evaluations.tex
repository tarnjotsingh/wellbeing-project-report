\section{Evaluations}

This section aims to be a reflection on the work that has been carried out over the course of the entirety of this project.

\subsection{Application evaluation}

\subsubsection{Successes of the application}
Considering the complexities involved with this project, what has been implemented has been a success.
The project has managed to cover the main groundwork with regards to both the server and client side of the overall 
software solution.


\subsubsection*{Scalability}
One of the main requirements for the system was for it to support future scalability.
Completely separating both the front and and back end components allows each layer to consist of multiple instances of the same application as 
a form of load balancing.

Adopting a session-less approach, with Oauth2 and JWT, means that many servers can be deployed at once without having to share user sessions.
Having no sessions stored server side removes scalability issues as the memory capacity of a server will end up being a limiting factor.
Tokens are just issued by an authorization server and are stored client side instead.
It is also much easier to perform horizontal scaling as no data needs to be shared between each server instance.
Another benefit is that there is no recovery required if a server needs to be restarted.

\subsubsection*{Security}
An emphasis was made on security for the application so time that could have been spent on features was spent on making the application more secure.
Developing a secure set of API endpoints, from the ground up, specifically for the student wellbeing application, has been a great success.
The problem was that this API is no good in a real world environment if any type of user can have access to resources that should be 
The developer was able to successfully incorporate their own Oauth2 implementation to provide session-less authentication where a client is given
a Json Web Token to access restricted resources.
Any registered user must provide valid credentials before they are issued an access token and will only be valid for a limited period of time.
Tokens that are issued have an associated token secret which only the server will know of, this is another method that is used to ensure requests are 
being made with a legitimate token.

\subsubsection*{Usability}
The survey creation tool, that has been fully implemented, managed to pass the acceptance testing stage which serves as proof of 
the sound functionality.
Even though there was no upfront design created for the client application, what has been developed is clearly presented and logically laid
out.
The components that make up each of the pages have been created in a way that ensures they behave the way users expect them to.
It would really frustrate the user if the application had not been updating other page components based on their own interactions.
Utilising Twitter's Bootstrap library has allowed the application to have a reactive layout; adapting to the screen size of the device being used.
In the absence of any mobile native application implementation, users can still access and use the web application as intended.

\subsubsection{Features not implemented}
Unfortunately due to the time constraints involved, the developer has not been able to implement a number of feature.
Most of these features related to the student usage of the application have not been implemented.

\subsubsection*{User registration}
Being able to register new users to the system is obviously an important part to the overall application. 
Before being able to handle user registrations, a secure method of authenticating users was considered a bigger priority.  
At the moment, the current implementation automatically \emph{registers} a test user when the user information in the database is detected to be 
empty.
With more time, a set of user registration API endpoints would have been created to support this functionality as the main groundwork has already 
been achieved.

\subsubsection*{Survey response tool}
The other major component of the application that was not included as a part of this release is the ability to respond to published surveys.
This is mostly because of the complexities involved with the survey creation tool and it did not make sense to start with a survey response tool
while there was no way to create them in the first place.

\subsubsection*{Android application}
The initial requirements stated that an android application needs to be provided as a part of the software solution.
A design decision was made to use the REACTjs library for developing the clint side application, the main motivation for this was that it would a 
lot easier to convert the existing code into REACT Native code which is used to create native applications for both Android and IOS.
If more time was available, then being able to learn how to convert the code and actually create such an application would have been possible.
The best alternative achieved that was achieved was the web application being fully usable on smart phones due to the taking advantage of
the Bootstrap library.


\subsubsection{Future considerations}


%--------------------------------------------------An emphasis was made on security for the application so time that could have been spent on features was spent on making the application more secure.-----------------------------------------------------------%

\subsection{Social, legal \& ethical aspects}
Social, legal and ethical concerns need to be addressed before this system can be deployed for public use.

\subsubsection{Social}

Regarding social aspects, this application revolves around the mental health of students within a university environment.
The idea behind it is to provide resources to those individuals that find it difficult to engage in the initiative of going to their general practitioner
(GP) to get assessed.
This software solution is meant to be a way to give users the anonymity that they may need to participate in something that can help improve 
various 
Hopefully it will allow those attending university to improve results in assessments throughout the year and achieve a good degree level.
Students can just use the application in a more informal way, utilising it to track the overall wellbeing and be able to compare it to other peers that
they interact with.


\subsubsection{Legal}
Personally identifiable information, of users, will be stored in datbases for this application to function.
Such identifiable information will include names, email addresses and data collected related to that person's wellbeing status.
Under the General Data Protection Regulation (GDPR), such information needs to be processed lawfully, fairly and in a transparent manner \cite{unihighlands2018sevenprinciples}.
The data itself must only be collected and used for legitimate purposes, in the case of this application it should only be used to calculate wellbeing 
scores and have that data saved in a user created account.
Data that is associated with a particular user needs to be removed from all parts of the system once the data is no longer needed or if the user makes
an explicit request.
Extra precautions should be taken to ensure that breaches do not occur to the database system and have that data leaked into the public domain.

This also can be paired with the Data Protection Act (1998) where there are eight conditions to comply when storing data.
These eight conditions include points that include keeping data secure, relevant and accurate for the relevant amount of time. 
Ensuring that data is processed in a lawful manner is also something covered under the act.

Before deploying the application into a full production environment where users will be able to use it, a full report will need to be generated that
outlines how well the overall system complies with these laws.
The involvement of data protection officers (DPO) is a fundamental part of GDPR. 
They ensure that compliance is met with the requirements set out by GDPR.

\subsubsection{Ethical}
No surveys were conducted as a part of this project, so no ethics approval was needed for this project.
In the future when this prototype is being developed further into a system suitable for a production environment, ethics approval will be requested
to gather user feedback about topics that pertain to the initial problem statement.

%-------------------------------------------------------------------------------------------------------------%
\subsection{Developer performance}

\subsubsection{Challenges}

\subsubsection{Strengths}

\subsubsection{Skills gained}