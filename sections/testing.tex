\section{Testing}

To ensure that each component of the application functioned as expected, testing of the entire system is required.
This section will mainly cover user acceptance testing scenarios of the currently implemented features in the software solution.
Potential customers can write acceptance tests to determine if the system is behaving correctly \cite{miller2001acceptance} and 
inherently tests all layers that form the overall architecture.
Acceptance testing also allows the developer to understand if the initial requirements laid out have been met or not.

As of the writing of this report, only the survey management side of the application has been mostly implemented.
This means that, in the application's current state, it behaves more like a generic survey creation tool than something that is specialised
for student wellbeing.
For this reason it is acceptable to form a small group of students from the university to participate in the acceptance testing.
They have been tasked with using the system normally but will also attempt to find faults and cause failures.

\clearpage
\subsubsection{Login page}

 \begin{longtable}{|p{.05\textwidth}|p{.15\textwidth}|p{.15\textwidth}|p{.15\textwidth}|p{.15\textwidth}|p{.05\textwidth}|p{.11\textwidth}|}
  \hline
  ID          &   Test name            & Test                                                                                           & Acceptance \par criteria                                                                                                           & Actual \par outcome                                                                                                                                        & Pass/\par Fail & Comments\\
  \hline\hline                                                                                                                             
  LP1 & Login with invalid credentials & The user is presented with an error                                                            & The user is not redirected to another page. \par Username and password fields are emptied. \par An error is presented to the user. & The system told the user that their credentials were bad. \par System reset username and password fields for them to fill in their credentials again.      & Pass           &         \\
  \hline                                                                                                                                   
  LP2 & Login with no credentials      & The user is presented with an error                                                            & The user is not redirected to another page. \par Username and password fields are emptied. \par An error is presented to the user. & The system told the user that their credentials were bad. \par System reset username and password fields for them to fill in their credentials again.      & Pass           &         \\
  \hline
  LP3 & Login with valid credentials   & The user successfully logs in with the provided credentials and is redirected to the dashboard & The user is able to successfully log into the application with their credentials and is redirected to the dashboard.               & The system accepted their credentails and redirected the user to the dashboard.                                                                            & Pass           &         \\
  \hline
  LP4 & Logout of the system           & Use the logout button to revoke current authentication and be redirected to the login page     & The user is sent back to the login page and cannot go back to the dashboard                                                        & User is sent back to the login page and cannot access any other part of the system                                                                         & Pass           &         \\     
  \hline
\end{longtable}

\subsubsection*{Comments}
Users were satisfied with the overall implementation of the login page and were unable to find any major flaws.
They also appreciated the clean presentation and how credential autocomplete was also compatible with it.

\subsubsection{Survey list page}

\begin{longtable}{|p{.05\textwidth}|p{.15\textwidth}|p{.15\textwidth}|p{.15\textwidth}|p{.15\textwidth}|p{.05\textwidth}|p{.11\textwidth}|}
  \hline
  ID  & Test name                                       & Test                                                                                                                                                                  & Acceptance \par criteria                                                    & Actual \par outcome                                                                                                                 & Pass/\par Fail & Comments\\
  \hline\hline                                                                                                                          
  SL1 & User can view list of surveys they have created & Click on survey manager button from dashboard and user should be redirected to the survey manager page where they will see a list of surveys they have created        & User is redirected to the survey manager page and can see all their surveys & User was successfully transferred to the survey manager page and was able to see all their surveys                                  & Pass           &         \\ 
  \hline
  SL2 & Add a new survey                                & The user is redirected the Add Survey page                                                                                                                            & The user is successfully redirected to the Add Survey page                  & The user was redirected to the Add Survey page where they could entier details for the question name and description before saving. & Pass           &         \\
  \hline                                                                                                                                                                                                                                 
  SL3 & Edit an existing survey                         & The user is redirected to the Edit Survey page                                                                                                                        & The user is successfully redirected to the Edit Survey page                 & The user was redirected to the Edit Survey page where they could alter survey details                                               & Pass           &         \\
  \hline
  SL4 & Displaying delete prompt                        & Clicking the delete button against the survey created in SL2 and making sure a confirmation prompt is shown                                                           & Delete survey confirmation prompt shown                                     & The confirmation prompt for deleting a survey is shown                                                                              & Pass           &          \\ 
  \hline
  SL5 & Delete an existing survey                       & Following on from SL4 confirm deletion for the survey created in SL2                                                                                                  & The survey created in test SL2 is removed from the survey list              & The survey created in test SL2 is no longer present in the survey list                                                              & Pass           &          \\
  \hline
\end{longtable}

\subsubsection*{Comments}
Overall, users were happy with the way the survey list page has been implemented.
The noted how the page was presented in a very clear manner and were easily able to add, edit and delete surveys. 
  
\subsubsection{Survey edit page}


\begin{longtable}{|p{.05\textwidth}|p{.15\textwidth}|p{.15\textwidth}|p{.15\textwidth}|p{.15\textwidth}|p{.05\textwidth}|p{.11\textwidth}|}
  \hline
  ID  & Test name                          & Test                                                                                                                                          & Acceptance \par criteria                                                                                                                                                               & Actual \par outcome                                                                                                                                                       & Pass/\par Fail & Comments\\
  \hline\hline                                                                                                                                                                                                                                                                                                    
  SE1 & Add a new survey                   & Click the Add Survey button, provide a name with description and save it                                                                      & User is sent back to the survey list with the new survey they added now present in the list                                                                                            & User was sent back to the survey list page and their survey was added to the list                                                                                         & Pass           &         \\ 
  \hline                                                                                                                  
  SE2 & Open survey editor for a survey    & Click on the edit button for the survey added in SE1 and user will be redirected to the survey edit page with all the survey details present  & User is sent to the survey edit page with all the survey details for the survey added in SE1 prsent                                                                                    & User was redirected to the survey edit page with all the fields filled with the details provided when creating the survey added in SE1                                    & Pass           &         \\
  \hline                              
  SE3 & Change the survey name             & Change the survey name for the survey created in SE1                                                                                          & User changes the survey name and presses the save button. User is then redirected back to the survey list page with the new survey name applied to the edited survey                   & User applied a new named to the survey and submitted the changes. They were sent back to the survey list page where the new name applied was reflected in the list        & Pass           &         \\  
  \hline                      
  SE4 & Change the survey description      & Change the survey description for the survey created in SE1                                                                                   & User changes the survey description and presses the save button. User is redirected back to the survey list page with the new survey description applied to the survey                 & User applied a new description to the survey and submitted the changes. They were sent back to the survey list page where the new name applied as reflected in the list   & Pass           &         \\
  \hline                        
  SE5 & Change survey name and description & Change both the survey name and description for the survey created in SE1                                                                     & User changes survey name and descrption and presses the save button. User is redirected back to the survey list page with both values being changed to the ones the user had submitted & User changed both name and desciption for the survey and submitted. They were sent back to the survey list page where the new values were shown against the edited survey & Pass           &         \\   
  \hline     
\end{longtable}

\subsubsection*{Comments}
Users encountered no problems with editing surveys, was very straightforward with everything being very clearly presented.

\subsubsection{Survey question editing}

\begin{longtable}{|p{.05\textwidth}|p{.15\textwidth}|p{.15\textwidth}|p{.15\textwidth}|p{.15\textwidth}|p{.05\textwidth}|p{.11\textwidth}|}
  \hline
  ID  & Test name                          & Test                                                                                                                                          & Acceptance \par criteria                                                                                       & Actual \par outcome                                                                                                                                                     & Pass/\par Fail & Comments                                                                   \\
  \hline\hline                                                                                                                                                                                                                                                                              
  QE1 & Add a new survey                   & Click the Add Survey button, provide a name with description and save it                                                                      & User is sent back to the survey list with the new survey they added now present in the list                    & User was sent back to the survey list page and their survey was added to the list                                                                                       & Pass           &                                                                            \\ 
  \hline                                                                                            
  QE2 & Open survey editor for a survey    & Click on the edit button for the survey added in SE1 and user will be redirected to the survey edit page with all the survey details present  & User is sent to the survey edit page with all the survey details for the survey added in SE1 prsent            & User was redirected to the survey edit page with all the fields filled with the details provided when creating the survey added in SE1                                  & Pass           &                                                                            \\
  \hline        
  QE3 & Add a new question with no choices & Attempt to add a new question with no question choices to the survey                                                                          & The question with no choices is added to the survey and can be viewed in the question list                     & The question with no choices is added to the survey and can be viewd in the questions list on the same page                                                             & Pass           & User feels this should not be allowed to happen                            \\
  \hline  
  QE4 & Add an empty question              & Attempt to add a new question with no name or choices                                                                                         & Empty question is not submitted or added to the question list                                                  & Question was not added to the list and refreshing the page did not load it in either                                                                                    & Pass           & User believes an error should be shown to them as more effective feedback  \\               
  \hline  
  QE5 & Add a question with choices        & Attempt to add a new question with at least one choices                                                                                       & The question is submitted and the question, with choices, is displayed in the list                             & The question and its choices were added to the question list and refreshing the page loaded them in again                                                               & Pass           &                                                                            \\
  \hline  
  QE6 & Edit a question choice             & Edit a question choice from the question added in QE5                                                                                         & Change a value in one of the question choices and submit. Value in the question list should also be updated    & The question choice that was edited was successfully submitted and value in the question list was updated. Force reload confirmed value was submitted to server         & Pass           &                                                                            \\        
  \hline  
  QE7 & Add an empty question choice       & Add a new question choice to question but leave one or more of the fields blank                                                               & The question choice with one or more empty fields is not added to the questions or listed in the question list & Submitting a question choice with one or more empty fields led to an exception thrown, crashing the client                                                              & Fail           & Error handling not present for this test scenario                          \\
  \hline
  QE8 & Delete an existing question choice & Delete a question choice that already exists under a question                                                                                 & Question is deleted from the question and is no longer listed in the question list                             & Question is successfully deleted and has been removed from the question list. Reloading the page confirmed that it was no longer being loaded from the server           & Pass           &                                                                            \\
  \hline    
\end{longtable}
 
\subsubsection*{Comments}
For adding and editing questions to the survey, users found it intuitive while maintaining simplicity of use.
When adding a new question without any choices, a couple users noted that they did not believe this should be allowed by the system.
They believed that an error message should be presented to the user telling them they need at least two question choices.
An attempt was made by more than one user to add a new question with no question name but include choices. 
The outcome for this attempt was that nothing happened and the question add window was closed.
Feedback for this particular behaviour was that there should be some sort of feedback to the user to confirm that the question was not added
or that they performed and illegal action.
Question editing raised one system failure when attempting to add a question with one or more empty field. 
Users who had encountered it strongly suggested this needs to be fixed as they found they needed to go back to the login page and navigate 
back to the survey edit page.


\subsubsection{Further feedback}

\subsubsection*{Login page}
There was confusion with why users had to use a provided set of credentials to log into the system as they wanted to be able to register a new account.
This is due to there not being enough development time to implement such a feature.

\subsubsection*{Survey list page}
One feature that was suggested, to be included in the future, is the ability to categorise surveys based on a particular topic or keyword.

\subsubsection*{Survey question editing}
One user suggested that questions should be editable directly from the question list rather than having to use a window.
This suggestion was actually a desire for the developer but was not implemented due to the complexities involved.

\subsubsection*{Testing conclusions}
The use of acceptance testing has proven to be very successful.
In software development not all scenarios are always considered and this is apparent with the tests users had carried out due to the one failure 
found.
Feedback provided by the users in the test group has been very valuable due to the type of quality of life improvements that would benefit users.
When the number of survey becomes large enough, it would be really difficult for users to sort through. 
Though this was never in the general scope of the project, the developer can see the potential value this has.
Originally when developing the question lists, there was a desire to have all the questions be directly editable from the list itself rather than
having to open a new window. 
The developer agrees that this could provide a better experience to users however had to resort to the currently implemented solution due to the 
large complexities encountered during development.

Considering that not all features have been implemented yet, the ones that have have proven to be fit for purpose.
Users were overall pleased with the performance and presentation that was provided by the application.