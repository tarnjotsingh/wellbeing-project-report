\section{Solution Approach}
The software solution is going to be a web application that connects to some sort of data storage.
In the modern web, there are many different approaches developers can take for creating scalable web applications.
An \textit{n-layered} architecture approach will be taken as this can provide the most flexibility for future scalability 
due to how vertical (more compute resources) or horizontal (more instances) scaling can be performed at each layer.

The 3 layer approach, typically used in industry, will be used for this project.
A presentation layer, application layer and data layer will therefore have to be created.

\subsection{Presentation layer}
Talking about how ReactJS is going to be used.
Also mention how React is just a library, it should be easy to port over the code to Android to build a native
android application that is consistent in design with the main web application.


\subsection{Application layer}
Functionality of the presentation layer will be dependant on the requests made to the application layer.
An application programming interface (API) will need to be exposed to provide that desired functionality.
As a modern web application, a set of RESTful API endpoints will be created.
The Java Spring Framework will be used to create a library of RESTful API endpoints for managing the contents of a connected database.


\subsubsection{Gradle}
Gradle is an open-source build automation tool for Java, just like Apache Maven, and was initially release in 2007 \cite{muschko2014gradle}.
Considering the vast number of dependencies available for Java, a built tool such as Gradle makes it much easier to bring in
a variety of different open-source components.
The idea behind Gradle was to take the best aspects of existing tools, such as Apache Ant and Maven, and improve on them.
Gradle is JVM native which allows for developers to write plugins and scripts with Java or Groovy; whichever is more convenient. 

\subsubsection{Springboot}

Stuff to do with springboot 

\subsubsection{REST API endpoints}

Stuff to do with rest api endpoints.


\subsection{Data layer}

Talk about the use of a relational database and include details about the initial approach to the schema.