\section{Conclusion and personal reflections}

Overall, I am quite satisfied with what has been produced for this project.
Though not all features were able to be implemented, some of the main technical challenges were overcome to create an application that is easy 
to use, is secure and allows for future scalability if needed.
In the current iteration, as of the writing of this report, the solution is more generic survey editor than something specialised for student wellbeing.
This is acceptable, in my opinion, as the fundamental features that form the building blocks of a generic survey application are required 
before more tailored features can be incorporated to make it targeted towards student wellbeing.
I find a sense of satisfaction with the current implementation of the survey creation and editing tool as this is the first time I have used JavaScript
(more specifically REACT) to create such a user interface.  
If I started off knowing the core fundamentals of all the technologies used in the project, then the time spent on researching technologies would have
been used for feature development instead.
Progress did not occur at quick enough pace until about late March at which point focus had to be spent on polishing what had been implemented instead 
of including more features. 
For what it is worth, developing this application has provided me with the fundamental knowledge that I wished I could have started off with.
Given more time many of the missing features could have been implemented, most notably the survey response page that students could have used.
User registration, a fundamental feature for web apps, would now be trivial to implement as the complexities of user authentication and token 
generation have been overcome.

The user interface itself has proven to be a very clean and effective implementation, only requiring two pages to be implemented.
A concious decision to reduce the number of interfaces the end user would be exposed to, was an essential part of the design philosophy.
Every design choice was taken from the perspective of a potential user to ensure that application is not frustrating or cumbersome to use; ensuring 
form and function are together.
This claim is backed up by the comments made during the acceptance testing was carried out (section \ref{testingsection}).
Acceptance testing that was carried out proved to be an effective measure in the absence of actual code coverage using unit tests.
The lack of these testing cases mainly came down to how I was having to learn a lot of the technologies while I was building each component.
This meant that development did not take a test driven approach (which was desired) because I did not know what type of tests to write before coding.
It seems to be a very typical thing when I am developing with new technologies but should not be something that is carried forward.
This very situation has led to me developing a strong interest to learn proper TDD (Test Driven Development) techniques to use in future projects. 

This project has proven to be an effective exercise in bringing together concepts taught in modules I have taken throughout my time in the university.
Coming into the university as someone who had no idea how to write code, to incorporating a range of technologies for a project that contains a number 
complexities has been a real eye opener to how much I have managed to learn.
This project alone has exposed me to relevant development stacks that are widely used across industry and has taught me how to create a competent 
web application.
Moving forward I am confident that my passion for technology and (now) development will allow me to undertake more challenging projects to hopefully
make something that can help shape the future in some way.

Many learning opportunities have arisen while working on this project.
From choosing to work on the problem, architecting a solution then following through with the implementation, there was always something to learn.
After overcoming initial issues with focusing on the project, I believe that I was able to manage my time well to deliver an initial version of the
software solution.
Though not all the functionalities specified in the requirements have been implemented, I was still able to create my own REST API and utilise that 
in the client to make functioning survey management tool.
Given more development time I am confident that I can finish off development of all the necessary features to make this software solution have a real 
impact on the overall wellbeing of students in university environments.