\section{Literature Review}

In the early stages of this project research was conducted of existing literature and technologies to grasp a better understanding of the problem and what would be the best approach to solve it.
This section will cover the main areas that this project will focus on.

\subsection{Wellbeing of university students}

Since this report revolves around the topic of student wellbeing, it is important to establish a definition for the term "student wellbeing" to correctly approach the problem.
It is defined as a sustainable state of positive mood and attitude, resilience, and satisfaction with self, relationships and experiences at school \cite{noble2008scoping}.
The part referring to a sustainable state is important to focus on. All humans endure fluctuations their emotional state within their lives, but a real problem occurs when their emotional state is not sustainable.
There are many factors that can play into poor mental health. Stress tends to be one of the main issues. 
Stress arouses feelings of fear, incompetence, uselessness, anger and guilt \cite{turunen2014indoor}, it can be expected that this will lead to behavioural changes. 
To some, these attributes that stem from stress can act as a form of motivation, however not all are able to do such a thing.
Those that are unable to cope with stresses from university studies will, unfortunately, struggle to focus on assignments and revision for exams.
A study by HSBC, found a correlation between disadvantaged social circumstances with increased health risks \cite{currie2009social}.
While talking about students, it is important to recognise that there may be deeper stemmed issues within the lives of certain students that may affect them when entering higher education.

Most disorders that pertain to mental health first start developing between the ages of 15 and 20 years of age \cite{kessler2005lifetime}.
By taking this fact into account, 15\% to 20\% of individuals who fall into this category (youth \cite{youth2017definition}) are affect by some sort of mental disorder; the majority whom suffer from an anxiety order. 
The study, conducted by The university Of British Columbia, therefore came an approximate figure of 140,000 children and youth who suffer from a mental disorder. 
UCAS have reported that, on average across the United Kingdom, that 27\% of people aged 18 years old had been accepted into universities \cite{ucas_2018}.
This percentage makes up a total of 353,960 students. 
After applying the 15\% estimate, there are likely around 53,000 students that are starting their university life, in 2018, with undiagnosed mental health problems.

Stigmas/stereotypes associated with having such a condition are accepted in western societies \cite{corrigan2002paradox}.
By acknowledging this, one can understand how it is difficult for these individuals to engage in constructive discussions.
Anonymising feedback for those that wish to get help without anybody judging would be a good place to start. 

\subsection{What makes a good survey creation tool?}

The main objective of this project is to provide the necessary tools for health experts to curate surveys for students to respond to.
Data from completed surveys can be used to direct students to appropriate resources depending on choices for each question.


\subsection{Web-app architectures}
In the world of software engineering, there are many approaches to solve a single problem.
Throughout the years, along with the evolution of technology, approaches have changed to ensure that products are
can be developed and maintained in a manner and time that allows businesses to adapt and react to ever changing markets. 
One of the main concerns of a business is to grow [citation], this is why having an architecture in place that is scalable is
critical to maintain a competitive edge and meet the demands of customers.

There are common architecture approaches used when developing a web application.
A recent article by Microsoft listed approaches typically taken as of 2019. 
Monolithic and n-layer approaches are the typical avenues taken by software developers in recent years \cite{ardalis_common}.
An architecture model is chosen based on what is required and how quickly an application needs to be developed.

\subsubsection{Monolithic}
Monolith applications is one that is entirely self-contained, in terms of its behaviour \cite{ardalis_common}.
It may interact with other services or data stores, but the core functionality is shipped as a single unit.

The main motivation for wanting to deploy an application built this way is how it is easier to deploy \cite{namiot2014micro}.
Issues arise with this approach is when new features need to be added. The need to grow the feature set of an application likely means
more developers need to be recruited. A large, intertwined, codebase generally requires new recruits to spend a lot more time learning
how the existing codebase functions and understand how to correctly implement new code that does not compromise the existing stability.
A lot of communication will be required by a team, while developing features, to ensure compatibility.
This results in a lot of wasted time as it fundamentally compromises productivity and may result in poor quality code to meet deadlines.
In order to scale such an application, the only clear approach would be to run multiple instances of it behind a load balancer.
This is a very inflexible way to adapt to growing demand for a service.
Due to how multiple instances of the same application will be required, the resources required will also need to scale at the same rate.
For large companies, this is just un-feasible due to the costs for either buying or renting the resources. 

Ideally monolith application should only be used while developing initial prototypes. If they are to be deployed in production environments,
it should really only be for very small client bases that are unlikely to need extensive scaling.

\subsubsection{N-layer}

Modern web architectures, in production environments, typically adopt a three tier approach which is a client-server architecture approach.
This approach to software design allows presentation, data processing and data storage functionalities to be separated.
A result of this is that three distinct layers to the project can be worked on independently from one another; even separate teams for
each product layer. 

Less learning will be required for new recruits due to less code, compared with a monolith codebase, needing to be learned.
Scaling an application that utilises a layered architecture approach would be a lot easier to scale due to the independence of each layer.

\subsection{Web-app technologies}

\subsubsection{Presentation layer technologies}
A user interface for a web based application is found as part of the presentation layer.
This front end application will rely heavily on the need to send and receive requests from a server application (application layer) as 
it drives the core business functions.
As it is an application, technologies that allow for dynamic interactions with the web page are required.

HTML (Hyper Text Markup Language) with CSS (Cascading Style Sheets) are core technologies for building web pages \cite{w3c_html_css}.
Structure of a web page is defined by HTML whereas styling and layout is defined by the CSS used.
Web applications need to enable interactivity with the end user; HTML and CSS alone cannot achieve this.
Javascript is predominately used to provide dynamic behaviour and is based off the ECMAScript standard \cite{stefanov2010javascript}.
Through the large amount of support Javascript has had over the years, a number of frameworks and libraries have come about.
A focus will be made on three of the most popular tools for creating professional user interfaces; Angular, React and Vue \cite{stateofjs_2018}.

\subsubsection*{W3C Document Object Model (DOM)}
It is important to define what DOM is before discussing JavaScript frameworks and technologies due to the reliance on the manipulation of the DOM to 
create dynamic web pages that users can interact with.
DOM is a W3C standard that defines a language and platform-neutral application programming interface (API) for accessing, navigating, and 
manipulation of HTML and XML documents.
It defines the logical structure of documents and the way a document is accessed and manipulated \cite{wood1999programming}.
The main benefit of the DOM is the way it allows programmers to build documents, navigate its structure and manipulate elements on that page.
JavaScript allows web pages to become dynamic as it facilitates the, on the fly, manipulation of the DOM for a HTML document.

\subsubsection*{Angular}
Angular is a JavaScript MVC framework created by Google to properly architecture and maintain web applications.
It embraces extending HTML into a more expressive and readable format. 
Since Angular is built and maintained by Google, there is a large community available that developers can learn from.
Angular evaluates the markup of a HTML document only after it has been loaded into the DOM.
This approach brings about three key benefits.

As stated, Angular only starts evaluating a page after the main loading process has been completed.
The main advantage of this is that it allows developers to start implementing Angular code into existing pages.
Since Angular is completely rendered on the client-side, it removes the need for any server or build process. 
If code has been added into a HTML page, it can be opened up into a browser and have scripts be executed straight away.
Extensibility is another great feature of Angular where the option to create custom elements is available for developers.
Developers can also define how an element is rendered and assign behaviours to it; allowing custom libraries to be built \cite{jain2015angularjs}.

\subsubsection*{React} 
React was originally created by engineers at Facebook to solve the challenges involved when developing complex user interfaces with datasets 
that change over time \cite{gackenheimer2015introduction}. 
Unlike Angular, which is a framework, React is a library of tools that developers can use to create interactive pages.
Developers, over the years, have wanted to get away from directly interacting with the DOM due to how unfriendly it was for many.
The aim of React is to avoid this direct interaction with the DOM and instead deal with a virtual DOM that abstracts away from
the real one. This approach greatly simplifies the programming model involved while improving performance \cite{staff2016react}.

A big problem Facebook faced was keeping the user interface in sync with both the business logic and the state of the application.
Normally you would have to use a centralised event bus, by putting events into the queue or by having listeners for the event that would then go
ahead and make the necessary changes to a page. 
For programmers, this was overly cumbersome. Hence the desire to innovate to solve this problem.
React applications are able to automatically re-render themselves when there has been a change to the underlying data model.
A diff (comparison between before and after) is conducted to observe what has actually changed. The components that have exhibited a change are the
only ones to get re-rendered \cite{staff2016react}.

\subsubsection*{Vue}
Created by Evan You, Vue was created out of the need to quickly prototype a large user interface.
Rather than writing out a lot of repeated HTML, there was a desire to find a framework or tool that may already exists that solves the exact problem Evan had.
At that time Angular was widely used and React had only started. As great as these technologies were at the time, they were not flexible and lightweight
enough for fast prototyping of user interfaces. Though Vue started off as a tool for fast prototyping, it has matured over the
years to become something developers can use to build complex, scalable and reactive web pages \cite{filipova2016learning}.
The core library, for Vue, is focused on the view layer only, and is very easy to pick up and integrate with other libraries or existing projects \cite{koetsier2016evaluation}.
Similarities can be drawn to React where both use a virtual DOM and have components that can be reused.

https://vuejs.org/v2/guide/comparison.html comparisions between React and Vue from official Vue website.


\subsubsection{Application layer technologies}
The application layer is where all business logic would occur and facilitates features of an application.
Majority of the time, applications will be required to store data produced by end users. A data store, typically a database, is
used to do this.
Modern web architectures tend to rely on a three layer approach. 
Covering topics of:
\begin{enumerate}
    \item Java Springboot
    \item Nodejs
    \item Python
    \item RESTful APIs
    \item Microservice - might be worthwhile to look into breaking up user service from main service.
    \item Any other topics that pertain to how the backend has been built
\end{enumerate}

\subsubsection*{NodeJS Framework}
NodeJS Javascript framework stuff here.

\subsubsection*{Java Spring Framework}
Java Spring Framework/SpringBoot stuff here.

\subsubsection{Data layer technologies}
Section talking about different ways to store data and discussing the different database types
that will allow someone to store this data. Talk about relational and non-relational databases and
state why one is better than the other. You don't have to come to a conclusion here, just discuss
the different options.


\subsection{Security concerns}

Can do a section with regards to security concerns such as SQL Injection and other vulnerabilities.
Can reference this later when arguing why using the technologies that you have are the best choice.
Sub section to do with web security.
\begin{enumerate}
    \item HTTPS, SSL, TLS
    \item Oauth2 or just to do with signing in stuff.
    \item JWT
    \item Sessions
    \item Possibly explore other common vulnerabilities in the web
\end{enumerate}