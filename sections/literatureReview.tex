\section{Literature Review} \label{litreview}

In the early stages of this project research was conducted on existing literature and technologies to grasp a better understanding of the problem and what would be the best approach to solve it.
This section aims to cover the main areas that this project focuses around.

\subsection{Wellbeing of university students}

Student wellbeing is defined as a sustainable state of positive mood and attitude, resilience, and satisfaction with self, relationships and experiences at school \cite{noble2008scoping}.
The part referring to a sustainable state is essential to focus on. All humans endure fluctuations in their emotional state within their lives, but a real problem occurs when their emotional state is not sustainable.
Many factors can play in poor mental health. Stress tends to be one of the main issues. 
Stress arouses feelings of fear, incompetence, uselessness, anger and guilt \cite{turunen2014indoor}. One can expect that this can lead to behavioural changes. 
To some, these attributes that stem from stress can act as a form of motivation; however, not all can do such a thing.
Those that are unable to cope with stresses from university studies will, unfortunately, struggle to focus on assignments and revision for exams.
A study by HSBC found a correlation between disadvantaged social circumstances with increased health risks \cite{currie2009social}.
While talking about students, it is essential to recognise that there may be deeper stemmed issues within the lives of individual students that may affect them when entering higher education.

Most disorders that pertain to mental health first start developing between the ages of 15 and 20 years of age \cite{kessler2005lifetime}.
By taking this fact into account, 15\% to 20\% of individuals who fall into this category (youth \cite{youth2017definition}) are affected by some mental disorder; the majority who suffer from an anxiety order. 
The study, conducted by The University Of British Columbia, therefore came an approximate figure of 140,000 children and youth who have a mental disorder. 
UCAS has reported that, on average across the United Kingdom, 27\% of people aged 18 years old were accepted into universities \cite{ucas_2018} in 2018.
This percentage makes up a total of 353,960 students. 
After applying the 15\% estimate, there are likely around 53,000 students that are starting their university life, in 2018, with undiagnosed mental health problems.

Stigmas/stereotypes associated with having such a condition tend to be accepted in western societies \cite{corrigan2002paradox}.
By acknowledging this, one can understand how it is difficult for these individuals to engage in constructive discussions.
Anonymising feedback for those that wish to get help without anybody judging would be an excellent place to start. 

\subsection{Existing survey creation tools}
As this project aims to build a survey tool, it is important to look at existing solutions.
Existing solutions should give the developer a better understanding of what is expected out of such an application.

\subsubsection{Google forms}
A popular survey/form creation tool that is used by many people is Google Forms and is offered to users free of charge.
It allows users to create their own surveys and send them out via invitation links or by email.
Google has provides many options to the user that is creating the survey form, being able to choose from multiple choice, drop-down and linear-scale.
Responses to the survey are collected and the data can be presented to the creator in a multitude of different ways, depending on how they wish to
view it. 
The way the application has been developed means that it is reactive to the screen it is displayed on, being able to fit the screen based on its size \cite{google2015forms}.
Users can therefore use a user interface hat is consistent across all devices. 
Due to the free nature of this service, users can create as many surveys as they want with as many questions as they please.
There is no restriction on the number of users that can collaborate on surveys which can be very helpful when scaling the number of experts for curating 
a set of survey for students.

\subsubsection{Google opinion rewards}
Google opinion polls is an application specific to the Android plat form and is used by marketers to collection some basic data around a particular topic \cite{google2018opinionrewards}.
Polls that are sent to users are Usually very quick to fill out and require no form filling, simply just choosing between multiple available options.
To encourage users to complete the polls, users are rewarded with Google Play credit which can be used to buy anything on that store. 
One aspect of this application has that this project aims to incorporate is the Android application that can notify users when new surveys are available.
Inspiration can possibly be taken with the way this application handles the feature.

\subsubsection{Survey Monkey}
Survey Monkey is an internet application and hosting site that enables a person to develop a survey for use over the internet \cite{waclawski2012use}.
As is with other solutions, Survey Monkey is commonly used for market research but can also be applied to other areas such as health.
Unlike options, such as Google Forms, Survey Monkey includes pricing structures with certain features locked behind pay walls.

At a basic level, users can create an unlimited number of surveys but are restricted to only 10 questions for each of them with only 100 respondents.
This can potentially be an issue if it were to be used with the number of potential users in a university environment. 
Unless the university themselves are willing to pay for it, it may prove to be a barrier for usage.
Some level of branding could be required in the future to represent an organisation such as a university.
In this case costs are likely to increase by a substantial amount; likely in excess of £1200 just for a single premier account.

Another issue with this service is that only account holders included in the pricing plan are able to create the surveys.
This could make it hard to bring more health experts on board for curating surveys due to the costs involved.

\subsubsection{ZOHO}
ZOHO is an option that is similar to Survey Monkey in many ways; mostly due to the pricing strategy involved.
Some of the key features that are listed are the many templates, create surveys with drag and drop questions along with ability to include scoring \cite{zoho2018features}.
The scoring part of surveys is something that is of great interest in this project as it will be one of the key factors into being able to understand 
where students fall within a range.
Inspiration may be taken from this service for the development of this project.

\subsection{Web-app architectures}
In the world of software engineering, there are many approaches to solve a single problem.
Throughout the years, along with the evolution of technology, approaches have changed to ensure that products can be developed and maintained in a 
manner and time that allows businesses to adapt and react to ever-changing markets. 
One of the main concerns of a business is to grow, this is why having an architecture in place that is scalable is critical to maintain a competitive 
edge and meet the demands of customers.

There are conventional architecture approaches used when developing a web application.
A recent article by Microsoft listed approaches typically taken as of 2019. 
Monolithic and n-layer approaches are the typical avenues taken by software developers in recent years \cite{ardalis_common}.
An architecture model is chosen based on what is required and how quickly an application needs to be developed.

\subsubsection{Monolithic} \label{monolithic}
Monolith applications is one that is entirely self-contained, in terms of its behaviour \cite{ardalis_common}.
It may interact with other services or data stores, but the core functionality is shipped as a single unit.

The primary motivation for wanting to deploy an application built this way is how it is easier to deploy \cite{namiot2014micro}.
Issues that arise with this approach is when new features need to be incorporated. The need to grow the feature set of an application likely means
more developers need to be recruited. A large, intertwined, codebase generally requires recruits to spend a lot more time learning
how the existing codebase functions and understand how to correctly implement new code that does not compromise the existing stability.
Much communication will be required by a team, while developing features, to ensure compatibility.
It results in a lot of wasted time as it fundamentally compromises productivity and may result in poor quality code to meet deadlines.
To scale such an application, the only clear approach would be to run multiple instances of it behind a load balancer.
It is a very inflexible way to adapt to a growing demand for a service.
Due to how multiple instances of the same application are required, the resources required also need to scale at the same rate.
For large companies, this is just unfeasible due to the costs for either buying or renting the resources. 

Ideally, a monolithic approach should only be used while developing initial prototypes. If they are to be deployed in production environments,
it should only be for small client bases that are unlikely to need extensive scaling.

\subsubsection{N-layer} \label{nlayer}

Modern web architectures, in production environments, typically adopt a three-tier approach which is a client-server architecture approach.
This approach to software design allows presentation, data processing and data storage functionalities to be separated.
A result of this is that three distinct layers to the project can be worked on independently from one another; even separate teams for
each product layer. 

Less learning is required for new recruits due to less code, compared with a monolith codebase, needing to be learned.
Scaling an application that utilises a layered architecture approach would be a lot easier to scale due to the independence of each layer.

\subsection{Presentation layer (front-end) technologies}
A user interface for a web-based application is a part of the presentation layer.
This front end application relies heavily on the need to send and receive requests from a server application (application layer) as 
it drives the core business functions.
As it is an application, technologies that allow for dynamic interactions with the web page are required.

HTML (Hyper Text Markup Language) with CSS (Cascading Style Sheets) are core technologies for building web pages \cite{w3c_html_css}.
HTML defines the structure of a web page whereas styling and layout are defined by the CSS used.
Web applications need to enable interactivity with the end user; HTML and CSS alone cannot achieve this.
Javascript is predominately used to provide dynamic behaviour and is based on the ECMAScript standard \cite{stefanov2010javascript}.
Through a large amount of support, Javascript has had over the years, many frameworks and libraries have come about.
Three of the most popular tools for creating professional user interfaces, Angular, React and Vue \cite{stateofjs_2018} can be looked at for suitability for this project.

\subsubsection{W3C Document Object Model (DOM)}
It is important to define what DOM is before discussing JavaScript frameworks and technologies due to the reliance on the manipulation of the DOM to create dynamic web pages that users can interact with.
DOM is a W3C standard that defines a language and platform-neutral application programming interface (API) for accessing, navigating, and 
manipulation of HTML and XML documents.
It defines the logical structure of documents and the way a document is accessed and manipulated \cite{wood1999programming}.
The main benefit of the DOM is the way it allows programmers to build documents, navigate its structure and manipulate elements on that page.
JavaScript allows web pages to become dynamic as it facilitates the, on the fly, manipulation of the DOM for an HTML document.

\subsubsection{Angular}
Angular is a JavaScript MVC framework created by Google to properly architecture and maintain web applications.
It embraces extending HTML into a more expressive and readable format. 
Since Angular is built and maintained by Google, there is a large community available that developers can learn from.
Angular evaluates the markup of an HTML document only after it has loaded into the DOM.
This approach brings about three key benefits.

As stated, Angular only starts evaluating a page after the main loading process has completed.
The main advantage of this is that it allows developers to start implementing Angular code into existing pages.
Since Angular completely renders on the client-side, it removes the need for any server or build process. 
If code is added into an HTML page, it can be opened up into a browser and have scripts execute straight away.
Extensibility is another excellent feature of Angular where the option to create custom elements is available for developers.
Developers can also define how an element is rendered and assign behaviours to it; allowing custom libraries to be built \cite{jain2015angularjs}.

\subsubsection{React} 
React was created by engineers at Facebook to solve the challenges involved when developing complex user interfaces with datasets that change over time \cite{gackenheimer2015introduction}. 
Unlike Angular, which is a framework, React is a library of tools that developers can use to create interactive pages.
Developers, over the years, have wanted to get away from directly interacting with the DOM due to how unfriendly it was for many.
React aims to avoid this direct interaction with the DOM and instead deal with a virtual DOM that abstracts away from
the real one. This approach dramatically simplifies the programming model involved while improving performance \cite{staff2016react}.

A big problem Facebook faced was keeping the user interface in sync with both the business logic and the state of the application.
Usually, you would have to use a centralised event bus, by putting events into the queue or by having listeners for the event that would then go
ahead and make the necessary changes to a page. 
For programmers, this was overly cumbersome. Hence the desire to innovate to solve this problem.
React applications can automatically re-render themselves when there has been a change to the underlying data model.
A diff (comparison between before and after) is conducted to observe what has changed. The components that have exhibited a change are the
only ones to get re-rendered \cite{staff2016react}.
 
There is an extension to React is called React Native for creating native mobile applications. Unlike ReactJS, which is a JavaScript library,
React Native is a framework. 
This framework is based on ReactJS, but instead of targeting a browser application, it targets mobile platforms.
The idea behind this is to allow developers to write, fluid, native applications from the comfort and familiarity of an established library.
The code can be shared between platforms; making it easy to develop for both iOS and Android \cite{eisenman2015learning}.

\subsubsection{Vue}
Created by Evan You, Vue was created out of the need to quickly prototype a large user interface.
Rather than writing out a lot of repeated HTML, there was a desire to find a framework or tool that may already exist that solves the exact problem Evan had.
At that time Angular was widely used and React had only started. As great as these technologies were at the time, they were not flexible and lightweight
enough for fast prototyping of user interfaces. Though Vue began as a tool for fast prototyping, it has matured over the
years to become something developers can use to build complex, scalable and reactive web pages \cite{filipova2016learning}.
The core library, for Vue, is focused on the view layer only, and is very easy to pick up and integrate with other libraries or existing projects \cite{koetsier2016evaluation}.
Similarities can be drawn to React where both use a virtual DOM and have components that can be reused.

\subsection{Web services API} \label{web services}
An API (application programming interface) that allows two independent components of an application to communicate with one another.
SOAP and REST are two widely used technologies that are used when developing a web services API.

\subsubsection{Web 2.0}
Before being able to understand why SOAP and REST have come about, it is important for the reader to have a basic understanding of what is meant by Web 2.0 and how it.
The term has been around since 2005 but has been a controversial subject. It has mostly been controversial due to a lack of agreement on what exactly is meant
by the term \cite{constantinides2008web}. For this paper, the following definition from Techopedia will be used:
\begin{quotation}
    \noindent
    \textit{
        Web 2.0 is the name used to the describe the second generation of the world wide web, where it moved static HTML pages to a more interactive and 
        dynamic web experience.
        Web 2.0 is focused on the ability for people to collaborate and share information online via social media, blogging and Web-based
    }
    \cite{web2definition}. 
\end{quotation} 

\subsubsection{SOAP}

% Version 1.1 of SOAP specification, published on 8 May 2000, did not pass W3C recommendation status. It was only 24 June 2003 that it finally became a W3C recommendation;
% making it accepted as an industry standard.
SOAP (Simple Object Access Protocol) is a lightweight protocol for the exchange of information in a decentralised, distributed environment.
Though SOAP is a method of transferring data between a server and client, it is seen as a messaging protocol rather than something that defines
an API.
A point, which may matter to developers, is that XML is the only representation format supported. 
SOAP messages are sent as an \textit{envelope} and encompasses \textit{header} and \textit{body} elements \cite{gruschka2006protecting}\cite{box2000simple}. 
An example of how SOAP messages are structured, in XML, can be seen below. 

\begin{figure}[ht]
    \begin{lstlisting}[language=XML, numbers=none]
<?xml version = "1.0"?>
    <SOAP-ENV:Envelope xmlns:SOAP-ENV = "http://www.w3.org/2001/12/soap-envelope" 
     SOAP-ENV:encodingStyle = "http://www.w3.org/2001/12/soap-encoding">
        <SOAP-ENV:Header>
        ...
        </SOAP-ENV:Header>
        <SOAP-ENV:Body>      
        ...
            <SOAP-ENV:Fault>
            ...
            </SOAP-ENV:Fault>
        ...
        </SOAP-ENV:Body>
    </SOAP_ENV:Envelope>
    \end{lstlisting}
    \caption{Example SOAP message structure from TutorialsPoint.com}
\end{figure}

Web Services Description Language (WSDL) is typically used in combination with SOAP as a means to define the API. 
WSDL is an XML format for describing network services as a set of endpoints operating on messages containing either document-oriented or procedure-oriented
information \cite{christensen2001web}


\subsubsection{REST} \label{restliterature}
Representational State Transfer (REST) is a pattern of resource operations that has emerged as the (unofficial) standard for service design in Web 2.0 applications. Another way of describing REST is that it is an architectural style and a design for network-based
software architectures.
It is also essential to state that unlike SOAP, REST is an architectural style rather than a standardised protocol.
The term ``architectural style'' is defined as a context-free graph grammar.
The architecture itself is seen as the skeleton of an application and to be executable there needs to be a mapping between nodes
to entities in a given language \cite{le1998describing} .
In the case of REST, it would be to map exposed URL endpoints to a definition of what it needs to do in the codebase itself.

The simplicity involved with REST, along with its natural fit over HTTP, has contributed to its status as a method of choice for Web 2.0 applications
to expose their data. 
Resources in a RESTful service are both identified by and resolved with a URL \cite{battle2008bridging}.
The typical form you would see a URL in is: 

\begin{quotation}
    http://localhost:8080/Host/ApplicationPath/ResourceType/ResourceID
\end{quotation}

The core of any REST based design are CRUD (Create Read Update Delete) operations to deal with resources (data stored on the system).
Resources are a REST concept and are stateless and can be represented in a variety of forms such as HTML, XML or JSON.
The resources themselves can be manipulated through REST components. These components carry out requests and manipulate data through some kind
of uniform interface. HTTP, as an example interface, uses \textit{GET}, \textit{PUT}, \textit{POST} and \textit{DELETE} operations \cite{cdimascio2013restdesign}.
A great benefit of REST-based web design is the ability to use HTTP Headers to provide request context around each of theCRUD operations.
A request to a particular resource might result in HTML, XML, or JSON (compared to exclusively using XML with SOAP) depending on the desired
media type transmitted in the HTTP Accept header \cite{battle2008bridging}.


% Table to show a simple representation between CRUD to HTTP requests...
\begin{table} [ht]
    \begin{center}
        \begin{tabular}{ |l|l|l| }
            \hline
            CRUD Operation & HTTP Request & Response format\\
            \hline
            Create & POST & 201 CREATED\\ 
            Read & GET & Determined by request headers\\  
            Update & PUT & 200 OK \\   
            Delete & DELETE & 200 OK \\
            \hline
        \end{tabular}
        \caption{CRUD operation to HTTP request mapping}
    \end{center}           
\end{table}

\subsection{Application layer (back-end) technologies}
The application layer is where all business logic would occur and facilitates features of an application.
Majority of the time, applications need to store data produced by end users. A data store, typically a database, is
used to do this.
For a front end component of the application to request data from the application layer, an API(application programming interface) needs
to be provided. The API provided is what allows the client-side JavaScript application to ``plug'' into the server to provide the business
logic. 
Modern web applications (Web 2.0), as discussed in section \ref{web services}, adopt REST as the standard way of providing an API to use with client-side applications.
This section covers some of the technologies that are commonly used in the industry to create the application layer for a web application.

\subsubsection{NodeJS}
NodeJS is a server-side JavaScript environment. It is based on Google's runtime implementation called the ``V8'' engine.
V8 and Node are mostly implemented in C and C++, allowing for high-performance runtime with low memory usage.
The difference between V8 and Node is where V8 mainly supports JavaScript in the browser, Node aims to support long-running 
server processes \cite{tilkov2010node}.
As an asynchronous event-driven JavaScript runtime, Node is designed to build scalable network applications. 
Upon each connection, the callback is fired, but if there is no work to be done, Node will sleep; freeing resources \cite{nodejs2016}.

One of the critical advantages that NodeJS has over other solutions is the way it handles I/O.
Servers that implement multi-threading dedicate a thread to each browser connection in an attempt to prevent blocking calls being made \cite{frees2015place}.
Blocking is when the execution of additional JavaScript in the Node.js process must wait until a non-JavaScript operation completes \cite{nodejs2019blocking}. 
Managing resources can be challenging with this approach, especially when dealing with a large number of connections.
NodeJS only runs on a single thread and deals with the blocking issue by exposing asynchronous APIs.
Asynchronous operations are generally a ``fire and forget'' type of operation; the reason why it solves the issue of blocking.
Node Package Manager (NPM) has also been a vital tool in increasing the popularity of NodeJS.
In 2017, NPM had been reported to have over 350,000 packages \cite{linux2016stateofnpm}; demonstrating how active the community is.

A simple HTTP server is straightforward to create; only taking a few lines of code to get running. 
This type of simplicity that NodeJS provides a more productive environment for developers as there is little configuration to do;
especially when developing initial prototypes.

\begin{figure}[htb]
    \begin{lstlisting}[language=JavaScript, numbers=none]
    var http = require("http")

    /* Create a server that returns a basic html response
     * Runs on localhost on port 3000 */
    http.createServer(function(req, res){
        res.writeHead(200, {'Content-Type': 'text/html'});
        res.end("<!doctype html><html><body><p>Hello World!</p></body></html>");
    }).listen(3000, "127.0.0.1");
    \end{lstlisting}
    \caption{Simple NodeJS HTTP server listening on port 3000}
\end{figure}

Though NodeJS has significant benefits, many problems are surrounding it that prevent it from being used more widely in industry.
Due to the single-threaded nature, performance is poor when performing CPU intensive operations.
Several simple tasks (such as read/write database queries) can be queued in the background without blocking the main thread while
maintaining fast execution speed. 

Due to the asynchronous nature of JavaScript, NodeJS relies heavily on callbacks to know when an operation has completed.
The term "callback hell" is prominent in the JavaScript community due to the cumbersome nature of having to deal with them \cite{altexsoftnodejs}.
An example developers face is to do with error handling. The built-in try/catch mechanism does network with asynchronous callbacks \cite{gallaba2015don}.
Instead, workarounds are used to address issues such as this and are taken as a standard approach for performing such actions.
For developers accustomed to writing code that synchronously executes on multiple threads, it can be difficult for them to develop with asynchronous calls in mind.

\subsubsection{Java Spring Framework}
% Basic introduction
Java Spring is a framework designed to provide comprehensive infrastructure support for developing Java applications.
Infrastructure is handled by Spring, allowing developers to be more productive in application development.
The framework takes advantage of \textit{plain old java objects} (POJO) for building applications \cite{johnson2004spring}.
SpringBoot makes creating a standalone Spring-based application trivial. Minimal code and effort is required to create a SpringBoot 
project \cite{webb2013spring}.
Spring Initializer is a service provided by the creators as a way to quickly generate a SpringBoot application with a desired
configuration and is available at https://start.spring.io/.
Integrated development environments (IDEs), such as IDEA's IntelliJ, also provide in-built tools for generating such projects.

% Advantages
Just like with other technologies, such as NodeJS, creating a SpringBoot application is very easy. 
A simple server that listens on a specified port can be generated using the Initializer tool and be executed in little time.
Java has a mature set of dependencies available that are widely used by many Java developers and can be incorporated into a project with the help of dependency managers (i.e. Maven or Gradle).
Unlike JavaScript, Java supports multi-threading which can be very useful in operations that are compute heavy.
For certain operations that may cause blocking, Java has support for asynchronous executions though Java Futures.
The Spring framework has excellent support for interfacing with databases.
Users can create database entities as classes, allowing developers to and repository interfaces that removes the need for any SQL statements in the code itself.
Spring security is an entire component of the Spring framework that focuses on providing security features to a Spring application.

% Disadvantages
Lots of learning is involved when creating a Spring-based application.
Due to how Java is a statically-typed language, this can prove overwhelming for some developers who are more used to dynamic languages 
such as Python and Javascript.
Java is not as productive to work in compared to other, more straightforward, languages such as Python and NodeJS.
For quick prototyping, Java SpringBoot may not be the right choice for those that are not familiar with Spring already.

\subsection{Web security}
With any web application, security is a big concern that needs addressing.
This section aims to cover some of the main methods, that have been adopted by industry, of securing a web application.

\subsubsection{HTTPS, SSL and TLS}
The security extension to the HyperText Transfer Protocol (HTTP), is known as HTTPS.
It is used to establish a secure connection between two computers over a computer network and is used extensively used in the modern web.
Many services that require secure connections, such as banking and e-mail, rely on HTTPS.
HTTPS is based on the Transfer Layer Security (TLS) encrypted security protocol and supporting public key infrastructure (PKI).
The public key infrastructure is composed of thousands of trusted Certificate Authorities (CAs).
A Certificate Authority is an entity that issues digital certificates by associating a website public key with the respective domain name.

Transport Layer Security (TLS) has succeeded the Secure Sockets Layer (SSL) protocol.
TLS and SSL are cryptographic protocols that operate below the application layer and provides end-to-end cryptographic security for
applications protocols such as HTTPS and SMTP.
On a client's first connection, both the client and server perform a TLS handshake. During the handshake process, the server presents
the client with an X.509 digital certificate which is used to identify and authenticate the server to the client.
The client ensures that the certificate's identity matches the requested domain name, the certificate has not expired and the digital 
signature of the certificate is valid.
There is a public key that is associated with the certificate which is used by the client to share a session secret (session password)
with the server to establish the secure connection  \cite{durumeric2013analysis}\cite{thomas2000ssl}.
% Might reference more papers to ensure this section is sufficiently backed up.

\subsubsection{Oauth2 and JSON Web Tokens}
In the modern web, many applications rely on the use of Oauth2 based authentication to grant limited access to an HTTP service.
Oauth2 allows token-based authentication for requests to a secured resource.
These tokens are known as access tokens and are issued to clients by an authorisation server with the approval of the resource
owner.
An authorisation grant is used by a client to request a new access token to use in future requests.
Oauth2 allows for a range of grant types to support as many possible scenarios and use cases \cite{jones2015json}. 
Definitions for client names with an associated client secret can be defined only to allow requests from trusted clients.
Typically the client credentials are sent as basic auth (base 64 encoded) with user credentials are sent in the header 
of the request along with the grant type.

% Good examples of this are how you can sign in with Google or Facebook for certain applications.
JSON Web Tokens have seen industry-wide adoption as a means to provide a session-less authorisation method.
It serves as an alternative token type to issue to clients.
JWT is a compact claims representation format intended for space constrained environments such as HTTP authorisation headers.
There are three main parts to a JSON Web Token are the header, payload and a signature for the token which are all encoded
JSON objects. 
The signature of a JWT is referred to as the JSON Web Signature (JWS).  
A secret can be defined in the JWS, which can be shared by the client and server, to ensure the validity of the token.
The secret can be shared through another process before the JWT is shared with a client if required.
Depending on the implementation, secrets may not even need to be shared if a server provides an API option for checking the token itself.
A payload specifies user details and what authorities (permissions) are granted to that user. The payload also includes a
JTI (JWT Id) which ensures that an already issued JWT is not reissued (preventing replay) \cite{bradley2015json}.

Key advantages to using JWT \cite{raju2017jwt} over other solutions are that it is:
\begin{enumerate}
    \item Stateless:
    \begin{itemize}
        \item JWT is a self contained token which includes details about authentication and expiry time.
    \end{itemize}
    \item Portable:
    \begin{itemize}
        \item One token can be used with multiple backends.
    \end{itemize}
    \item Mobile friendly:
    \begin{itemize}
        \item No cookies are required which makes it great for use in mobile applications.
    \end{itemize} 
    \item Good performance:
    \begin{itemize}
        \item Network round-time is reduce which increases performance.
    \end{itemize} 
    \item Decoupled/decentralised:
    \begin{itemize}
        \item Tokens can be generated anywhere (e.g. Google or Facebook login)
        \item Authentication can happen on the resource server or be separated into a standalone service.
    \end{itemize} 
\end{enumerate}