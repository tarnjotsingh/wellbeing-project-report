\section{Problem Articulation and Objectives}
% Describe the problem, have plateau as is and plateau to be.
% Flow diagrams or something, describe what is wrong with the current process and propose a high level view of what the
% plateau/system to be is.

\subsection{Problem statement}

% Monitoring Student Well-being: The project will develop a database server application, a web-based application and an Android 
% application to captures student well-being information. The expert will use the interface to prepare a questionnaire for a survey.
% The questionnaire is uploaded into the database. The server application will push the questionnaire into the Android app in the 
% smart-phones of the students who have joined the well-being survey. Anonymous data is then collected and stored in the database. 
% In addition to collecting information from the students, the App may be used to direct students to sources of help if they are 
% in need. The system should be scalable as large amount of data will be collected and stored from a large number of students. 
% The system may also provide information back to the students to allow them to monitor their own well-being. For example, their 
% well-being at different times of the year/term, their well-being compared to other students at the University of Reading


% Introduction to the problem
Pressures from University life can cause a number of health issues in students.
In some cases students may experience stress attributed to not understanding topic and meeting deadlines.
Such an environment can also act as a catalyst for those that are already experiencing other issues outside of University.
Those that do not find themselves with the appropriate support may not have the confidence to arrange help sessions with their
General Practitioner (GP) or any on-campus support groups. 
In other situations, it may be helpful for students to assess their mental health situation and find ways to improve.

% Discuss the plateau-as-is/solution-as-is, identifying why it's not ideal.
Though many survey tools are available on the market, that are free to use, none have be tailored to address student wellbeing concerns.
Applications that are available, such as Google Surveys and Survey Monkey, are generic survey creation tools that can be shared with 
people via email or a link.
It makes sense that these online solutions are generic as they are aimed to cover as many use cases as possible.
This abstract approach to creating surveys is unlikely to be the ideal approach for specific issues around a single topic; such as
student well-being.
It is difficult to direct students to the appropriate resources, depending on the way they answered, as this type of functionality is not
provided by existing tools.
This project aims to create a web application that can be used across all devices and allow surveys to be created and pushed to students.
It would be useful, to students, to have an application that produce a score from completing a survey and have them directed to relevant
resources. The resources, that students can be directed to, should be configurable by the survey creator.





% \subsection{Technical specification \textbf{TEMPORARY FROM PID}}
% Server-side application
% Managing databases
% Providing an Application Programming Interface (API) to communicate with those underlying databases.
% Runs multiple small services for managing/processing survey/user data.

% Web-based application (Client-side)
% Allows for experts and students to log in and access the relevant features from a web browser environment.
% Use components that can be re-used when developing the android application which will allow for a more consistent experience for the user.

% Android application (Client-side)
% Allows for experts and students to log in and access the relevant features from an android application running on a mobile device. 
% Android application will mostly be targeted for running on mobile phones rather than tablets as there are fewer tablets than smartphones out there.

% Experts must be able to:
% Create surveys via some sort of interface on one of the client applications.
% Have a way to direct subscribed students to some sort of help depending on said student’s responses to a survey.
% Collect survey data and be able to visualise it. (Maybe there are libraries that already do this for you).

\subsection{Stakeholders}

Stakeholders mainly consist of: 

\begin{enumerate}
    \item students - hinders their performance in univeristy
    \item lecturers/university staff - potential concerns about how well students are coping
    \item mental health experts - Interest in helping the students using the tool to be made for this project.
    \item Think of someone else potentially...  
\end{enumerate}


\subsection{Project motivation}

Again just continue on from the stakeholders as they have active interests into the projet and the output of the project.

\subsection{Project constraints}