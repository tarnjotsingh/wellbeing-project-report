\section{Problem Articulation and Objectives}

\subsection{Problem description}
Describe the problem, have plateau as is and plateau to be.
Flow diagrams or something, describe what is wrong with the current process and propose a high level view of what the
plateau/system to be is.

\subsubsection{\textbf{TEMPORARY FROM PID}}
To tackle the problem correctly, the first thing I need to figure out is what exactly is student “well-being”. According to a website [ CITATION Wel13 \l 2057 ]  an individual is in good mental if that individual has positive self-esteem; feels/expresses a range of emotions; build and maintain good relationships; is engaged with the world around them; can live and work productively and more. As the application is going to be focused in the context of university students; it’s important that students have good well-being in order to carry out their studies. Feeling of wellbeing are fundamental to overall health of an individual, enabling them to successfully overcome difficulties and achieve what they want out of life.[ CITATION starJumpzWellBeing \l 2057 ].

[ CITATION Wel13 \l 2057 ] also list what can affect your mental well-being; such as bereavement, loneliness, money worries. What I gather from this information is that might be a way to group/theme surveys that an expert would like to create. Possibly have some sort of scoring system which help decide the right kind of help.

According to the brief provided in the staff proposed ideas document, the requirements are that there must be three distinct components; a server-side application; web-based application; android application. From experience from prior knowledge and some more research on github [ CITATION PiggyMetrics \l 2057 ], I felt that going with a microservices based architecture (for the server side) is better and provide an API (such as a REST API) that will allow the clients (web and android) to talk with the server/database. 

While researching into existing survey/opinion poll applications.
Google opinion polls [CITATION GoOp18 \l 2057 ] is an android application that pushes simple/quick to complete surveys to users. Upon completion, google awards you with some play store credits. It’s really good way to collect data as it provides the individual filling out the survey with an incentive to do it.
Google surveys [ CITATION GoSur18 \l 2057 ] is another service/product provided by google free of change for the user. It allows users to curate their own surveys and send them out via invitation links or by email. The responses to the survey are collected and the data can be presented to the creator. There are other more premium services such as Survey Monkey, ZOHO and Typeform that do the same basic thing, but offer some more elaborate features. Based on some of the more premium options, it could be an interesting way to include pricing models to generate some revenue out of the service. The user interfaces and pricing models of these more premium services could be a good source of inspiration for what I want to do. 

\subsection{Project objectives} 
Basically go over the main objectives that this project aims to solve.
Follow on from the problem description.
I mean you could probably see how the bullet point stuff are done on the example reports.


\subsubsection{\textbf{TEMPORARY FROM PID}}

For background research, the main things to plan for will be to research into existing survey solutions (as mentioned in section 2.1 of this document). This main reason for doing this is to gain a better idea of what already exists and what can be achieved with the time constraints. It also allows for improved clarity of what to aim for in the context of student well-being or just well-being in general. Looking around for what existing, open source, projects exist on repositories such as github is a good opportunity to start thinking about the development approach. From the research done in section 2.1, a microservices architecture may be an appropriate way to go for this project. The main deliverables for the research tasks will be this document along with some more detailed information in further documentation that will contribute to the final report.
Requirements and constraints will be an important part of the analysis and design stage. Figuring out what potential users that will act as the experts/students will value out of the system is important. It will identify any issues that would not have originally been considered. The requirements and constraints that have been gathered will feed into the high-level use cases. Those use-cases can feed into process flows for the system to see how data could be processed. The high-level architecture for the plateau-to-be can then be done, based on the requirements and use cases. A “just enough design” approach will be taken for the architecture design; not too detailed but has enough detail to show the reader what the initial approach will be. 
Prototypes may have already been developed in an aim to get focus, before getting to the development stage. Having multiple spikes (research tasks) for learning some of the technologies to incorporate will be needed. From those spikes, small writes ups of what was learnt can be the deliverables. It will also be important to get a start on defining the API, with the appropriate tools, based on the design work carried out. New client-side (web and android applications) prototypes are to be created and be plugged into dummy endpoints provided by the server application. Whether it should be tackled as vertical or horizontal slices will be decided later in more detailed design. Making iterative improvements will allow for regular deliverables to be produced; regular updates. 
Unit testing will be done alongside the main coding (hopefully from a TDD approach) which will allow for a certain level of code coverage to be maintained during development. Code coverage reports can be provided as another deliverable and proof of code quality. Further testing strategies to validate and verify the system.
Finally, the main project write-up will bring all the deliverables produced together. The final write-up should cover all aspects of the system produced.


\subsection{Problem statement}


\subsection{Technical specification \textbf{TEMPORARY FROM PID}}

Server-side application
Managing databases
Providing an Application Programming Interface (API) to communicate with those underlying databases.
Runs multiple small services for managing/processing survey/user data.

Web-based application (Client-side)
Allows for experts and students to log in and access the relevant features from a web browser environment.
Use components that can be re-used when developing the android application which will allow for a more consistent experience for the user.

Android application (Client-side)
Allows for experts and students to log in and access the relevant features from an android application running on a mobile device. 
Android application will mostly be targeted for running on mobile phones rather than tablets as there are fewer tablets than smartphones out there.

Experts must be able to:
Create surveys via some sort of interface on one of the client applications.
Have a way to direct subscribed students to some sort of help depending on said student’s responses to a survey.
Collect survey data and be able to visualise it. (Maybe there are libraries that already do this for you).

\subsection{Stakeholders}

Stakeholders mainly consit of: 

\begin{enumerate}
    \item students - hinders their performance in univeristy
    \item lecturers/university staff - potential concerns about how well students are coping
    \item mental health experts - Interest in helping the students using the tool to be made for this project.
    \item Think of someone else potentially...  
\end{enumerate}


\subsection{Project motivation}

Again just continue on from the stakeholders as they have active interests into the projet and the output of the project.

\subsection{Project constraints}